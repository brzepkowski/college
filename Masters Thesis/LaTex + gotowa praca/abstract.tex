\chapter{Abstract}
\thispagestyle{chapterBeginStyle}

Thanks to experiments carried out from 17th up to 20th century and the conclusions coming from them, it was possible to create new branch of physics - quantum physics. It is based upon the concept called wave-particle duality. 

This theory says, that when we are considering extremely small physical systems (using atomic scale), we cannot fully describe their behavior only in terms of particles, nor waves. We have to combine these two approaches to obtain model, which perfectly matches the empirical results. Einstein's words perfectly describe this phenomenon:

\begin{quote}
It seems as though we must use sometimes the one theory and sometimes the other, while at 
times we may use either. We are faced with a new kind of difficulty. We have two 
contradictory pictures of reality; separately neither of them fully explains the phenomena of 
light, but together they do.
\end{quote}

The most common and widely accepted (despite many controversies) way of explaining wave-particle duality is called the Copenhagen interpretation (introduced and promoted by Niels Bohr).

In 1982 Richard Feynman suggested, that we should use this contradictory with our intuition property of the Universe, to create a new device enabling simulation of physical systems. He said:

\begin{quote}
Trying to find a computer simulation of physics seems to me to be an 
excellent program to follow out [...] The real use of it would be with quantum 
mechanics [...] Nature isn’t classical [...] and if you want to make a simulation of Nature, you’d better make it quantum mechanical, and by golly it’s a wonderful problem, because it doesn’t look so easy.
\end{quote}

Nowadays we know, that such $quantum\ computer$ can be used not only for the purposes, for which Feynman proposed it, but also to solve any problem, that can be solved with a classical computer. Despite common opinion, that such machine would exceed any classical data processing device in solving every possible task, our current knowledge lets us think, that they would provide speedup only for a very specific scope of problems.

This work is structured as follows. In the first section of the "Introduction" chapter we will demonstrate fundamental theorems of functional analysis. In the next section, we will present elementary postulates of quantum mechanics and apply knowledge from the first section, to derive mathematical model describing the behavior of quantic entities. The first chapter will end with section, where we will introduce reader to the basics of classical computational complexity theory and basing on this knowledge, we will demonstrate computational complexity classes using quantum physics. In the second chapter we will demonstrate model used in the description of quantum computations. Third chapter will present three main areas of application of quantum computer – accordingly, quantum search, algorithms based on quantum Fourier transform and finally simulation of physical systems. In the fourth chapter we will summarize all results obtained in this work.

All algorithms created as part of this work were implemented on the IBM's quantum processors (IBM Quantum Experience) with the use of \textit{qiskit} python API. 


