\chapter{Summary}
\thispagestyle{chapterBeginStyle}

In the first chapter of this work we introduced all necessary information needed to understand basics of quantum computing. We showed, that we can understand functions as vectors and showed how complex and Hermitian conjugates are performed. Later we gave definition of vectors space and we were constantly expanding it with new properties until we received the definition of a Hilbert space. We explained such concepts as normalization, ortogonality and orthonormality of vectors. We introduced the idea of an operator and its eigenfunctions, eigenvectors and eigenvalues. In the second section of this chapter we introduced the widely used in quantum physics notation and presented elementary postulates of quantum mechanics. Finally, in the third section we introduced the reader to the basics of classical computational complexity theory and expanded this field with new classes, using quantum mechanics.

In the second chapter we defined basic concepts used in quantum computing, such as qubit and reversible computing. We presented different kinds of quantum gates (with single or multiple qubits). We showed how multiply controlled $C^n NOT$ gates can be implemented using different kinds of ancilla qubits. At the end of this chapter we presented sets of gates, which can be considered as universal (any computation that can be done using quantum circuits, can be done using only such gates).

In the third chapter we introduced three main areas, in which quantum computing can be used. We showed the idea behind the Grover's algorithm and presented in detail, how it can be implemented. Later, we defined the quantum Fourier transform and showed reasoning, which allows for implementation of QFT on quantum circuit. Afterwards, starting from the phase estimation algorithm we arrived at the Shor's algorithm and gave details, on how it can be implemented using quantum gates. At the end of this chapter we gave simple example, of how quantum computers can be used to simulate physical systems. 



